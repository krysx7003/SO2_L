\documentclass{article}
\usepackage{graphicx}
\usepackage{float}
\usepackage{titlesec}
\usepackage{datetime}
\usepackage{geometry}
\usepackage{minted}
\usepackage{placeins}
\usepackage{caption}
\usepackage[document]{ragged2e}
\usepackage[hidelinks]{hyperref}
\usepackage{enumitem}
\geometry{
 a4paper,
 left=25mm,
 top=25mm,
 }
\captionsetup{hypcap=false} 
\newdateformat{daymonthyear}{\THEDAY .\THEMONTH .\THEYEAR}
\title{
  \centering
  \includegraphics[width=\textwidth]{images/logo_PWr_kolor_poziom.png}\\
  \fontsize{28pt}{30pt}\selectfont Sprawozdanie 1\\
  }
\author{Krzysztof Zalewa}
\date{\daymonthyear\today}
\renewcommand*\contentsname{Spis treści}
\renewcommand{\figurename}{Rysunek}
\renewcommand{\listingscaption}{Skrypt}
\begin{document}
    \maketitle
    \pagebreak
    \tableofcontents
    \FloatBarrier
    \raggedright
    \section{Cel skryptu}
        W ramach zajęć laboratoryjnych należało wykonać dwa skrypty \linebreak
        1. Skrypt wypisuje 40 losowych liczb i zapisuje je do pliku A.txt następnie znaki 0 zastąpiane są O \linebreak
        2. Skrypt wykonuje operacje na strumieniach (minimum 3). Powstały strumineń powinien mieć realne zastosowanie.
        W moim przypadku tworzony jest 7 cyfrowy pin.
    \section{Opis skryptu}
      
        \begin{frame}
            \scriptsize
            \inputminted[
                style={vs},
                breaklines,
                breakanywhere, 
                linenos, 
                tabsize=4 
            ]{bash}{./lab3_1.sh}
            \vspace{1em}
            \captionof{listing}{}
            \label{lst:script_1}
        \end{frame}
      
        \begin{frame}
            \scriptsize
            \inputminted[
                style={vs},
                breaklines,
                breakanywhere, 
                linenos, 
                tabsize=4 
            ]{bash}{./lab3_2.sh}
            \vspace{1em}
            \captionof{listing}{}
            \label{lst:script_2}
        \end{frame}

    \section{Wnioski}
        W wyniku uruchomienia skryptu 1: \linebreak
        Plik A.txt zawiera ciąg 5532640758763844392042393771802658908476 \linebreak
        Plik B.txt zawiera ciąg 553264O758763844392O423937718O26589O8476 \linebreak

        W wyniku uruchomienia skryptu 2: \linebreak
        Plik Wyniki.txt zawiera ciąg 3061969 \linebreak

        Oznacza to że skrypty wykonują się poprawnie.

\end{document}